\documentclass[12pt, a4paper, onecolumn, oneside, final]{report}

%-------------------------------------------------------------------%
%
% Konfigurasi dokumen LaTeX untuk laporan tesis IF ITB
%
% @author Petra Barus
%
%-------------------------------------------------------------------%
%
% Berkas asli berasal dari Steven Lolong
%
%-------------------------------------------------------------------%

% Ukuran kertas
\special{papersize=210mm,297mm}

% Setting margin
\usepackage[top=3cm,bottom=2.5cm,left=4cm,right=2.5cm]{geometry}

\usepackage{mathptmx}

% Pengaturan bahasa
\usepackage[english]{babel}
\usepackage[figurename=Gambar ]{caption}
\usepackage[tablename=Tabel ]{caption}
\addto{\captionsenglish}{ %
    \renewcommand{\contentsname}{Daftar Isi}
    \renewcommand{\listfigurename}{Daftar Gambar}
    \renewcommand{\listtablename}{Daftar Tabel}
    \renewcommand{\appendixname}{Lampiran}
}

% Format citation
\usepackage[
    style=ieee,
]{biblatex}

\usepackage[utf8]{inputenc}
\usepackage{microtype}
\usepackage{makecell}
\usepackage{graphicx}
\usepackage{listings}
\usepackage{tabto}
\usepackage{comment}
\usepackage{amsmath}
\usepackage[labelfont=bf]{caption}	% Package dengan opsi untuk mempertebal label caption
\usepackage{enumitem}				
\usepackage{tocbibind}				% Package untuk memasukkan Daftar Pustaka ke dalam Daftar Isi
\usepackage{tocloft}				% Package untuk mengatur Daftar Isi, Daftar Gambar dan Daftar Tabel
\usepackage{float}					% Package untuk membantu penetapan lokasi gambar
\usepackage{indentfirst}			% Package untuk indentasi pada paragraf pertama
\usepackage[auto]{chappg}			% Package untuk penomoran halaman Bab-Halaman
\usepackage{titling}
\usepackage{blindtext}
\usepackage{sectsty}
\usepackage{chngcntr}
\usepackage{etoolbox}
\usepackage{hyperref}       		% Package untuk link di daftar isi.
\usepackage{titlesec}       		% Package Format judul
\usepackage{parskip}
\usepackage[htt]{hyphenat}
\usepackage{csquotes}

% Package untuk tabel
\usepackage{longtable}
\usepackage{booktabs}
\usepackage{multirow}
\usepackage{tabularx}

% Package untuk tambahan simbol matematika
\usepackage{amssymb}

% Line satu setengah spasi
\renewcommand{\baselinestretch}{1.5}

% Pengaturan tambahan untuk tabel
\renewcommand{\arraystretch}{1.1}
\setlength\heavyrulewidth{0.225ex}

% Setting judul
\titlespacing*{\chapter}{0pt}{-50pt}{10pt}
\chapterfont{\centering \Large}
\titleformat{\chapter}[display]
  {\Large\centering\bfseries}
  {\chaptertitlename\ \thechapter}{0pt}
    {\Large\bfseries\uppercase}

% Setting nomor pada subbsubsubbab
\setcounter{secnumdepth}{3}
\setcounter{tocdepth}{4}

\makeatletter
\setlength{\@fptop}{0pt}
\setlength{\@fpbot}{0pt plus 1fil}
\makeatother

% Counter untuk figure dan table.
\counterwithin{figure}{chapter}
\counterwithin{table}{chapter}

% Counter untuk penomoran halaman lanjut
\newcounter{savepage}

% Pengaturan caption
\captionsetup{labelsep=space}

% Pengaturan spasi untuk justify
\pretolerance=10000
\tolerance=2000 
\emergencystretch=10pt
%\tolerance=1
%\emergencystretch=10pt
%\hyphenpenalty=10000
%\exhyphenpenalty=10000

% Pengaturan untuk Daftar Rumus
\newcommand{\listequationsname}{Daftar Rumus}
\newlistof{myequations}{equ}{\listequationsname}
\newcommand{\myequations}[1]{%
\addcontentsline{equ}{myequations}{\protect\numberline{\theequation}#1}\par}
\setlength{\cftmyequationsnumwidth}{2.5em}

\renewcommand*{\theequation}{\thechapter.\arabic{equation}}

\renewcommand{\lstlistingname}{Algoritme}
\renewcommand{\lstlistlistingname}{Daftar \lstlistingname}


% Pengaturan untuk title Daftar Isi, Tabel, dan Gambar
\renewcommand{\cfttoctitlefont}{\hspace*{\fill}\Large\bfseries\MakeUppercase}
\renewcommand{\cftaftertoctitle}{\hspace*{\fill}}
\renewcommand{\cftlottitlefont}{\hspace*{\fill}\Large\bfseries\MakeUppercase}
\renewcommand{\cftafterlottitle}{\hspace*{\fill}}
\renewcommand{\cftloftitlefont}{\hspace*{\fill}\Large\bfseries\MakeUppercase}
\renewcommand{\cftafterloftitle}{\hspace*{\fill}} 
\renewcommand{\cftequtitlefont}{\hspace*{\fill}\Large\bfseries\MakeUppercase}
\renewcommand{\cftafterequtitle}{\hspace*{\fill}} 

\renewcommand{\cftchappresnum}{Bab }
\renewcommand{\cftchapaftersnum}{}
\renewcommand{\cftchapnumwidth}{3.7em}

\setlength{\cftbeforetoctitleskip}{-4em}
\setlength{\cftbeforeloftitleskip}{-4em}
\setlength{\cftbeforelottitleskip}{-4em}
\setlength{\cftbeforeequtitleskip}{-4em}

\newcolumntype{Y}{>{\centering\arraybackslash}X}

\lstset{frame=Tb,
  columns=fullflexible,
  basicstyle={\small\ttfamily},
  breaklines=true,
  breakatwhitespace=false,
  postbreak=\mbox{$\hookrightarrow$\space},
  tabsize=4
}

\renewcommand{\cftchapleader}{\cftdotfill{\cftdotsep}}

\cftsetpnumwidth{2em}

\DefineBibliographyStrings{english}{
    urlseen = {Waktu akses},
    url = {URL:},
    and = {dan},
    pages = {halaman},
    andothers = {dkk\adddot},
    techreport = {Dok. teknis},
    phdthesis = {Disertasi doktoral\adddot},
    in = {dalam}
}

\makeatletter

\makeatother

\bibliography{references}

\newcommand{\nim}{18270001}
\newcommand{\yearsidang}{1970}
\newcommand{\namapembimbing}{Dr. Apple Seed}
\newcommand{\nippembimbing}{19400101 196801 1 001}
% Kalau ada pembimbing kedua, isi ini
\newcommand{\namapembimbingkedua}{Orange Seed, B.Eng, M.Eng}
\newcommand{\nippembimbingkedua}{19400101 196101 2 004}

\begin{document}

    \title{Template Tugas Akhir Prodi STI dalam Bahasa Indonesia menggunakan \LaTeX}
    \date{1 Januari}
    \author{John Doe}

    \pagenumbering{roman}
    \setcounter{page}{0}
    
    \input{komponen-awal/cover}
    
    % comment out yang tidak digunakan
    \clearpage
\pagestyle{empty}

\begin{center}
\smallskip

    {\Large \bfseries Lembar Pengesahan}

    \MakeUppercase{\normalsize \bfseries \thetitle}
    \vfill

    \normalsize Tugas Akhir \\
    Program Studi: Sarjana Sistem dan Teknologi Informasi \\
    Sekolah Teknik Elektro dan Informatika \\
    Institut Teknologi Bandung \\
    \vfill

    \normalsize oleh :

    \normalsize \theauthor \\
    \normalsize NIM: \nim

    \vfill
    \normalsize \normalfont
    Telah disetujui dan disahkan sebagai Laporan Tugas Akhir \\
    di Bandung, pada tanggal \thedate{} \yearsidang{}.

    \vfill
    {\bfseries Pembimbing} \\
    \vfill
    \underline{\namapembimbing} \\
    NIP. \nippembimbing

\end{center}
\clearpage

    % \clearpage
\pagestyle{empty}

\begin{center}
\smallskip

    {\Large \bfseries Lembar Pengesahan}

    \MakeUppercase{\normalsize \bfseries \thetitle}
    \vfill

    \normalsize Tugas Akhir \\
    Program Studi: Sarjana Sistem dan Teknologi Informasi \\
    Sekolah Teknik Elektro dan Informatika \\
    Institut Teknologi Bandung \\
    \vfill

    \normalsize oleh :

    \normalsize \theauthor \\
    \normalsize NIM: \nim

    \vfill
    \normalsize \normalfont
    Telah disetujui dan disahkan sebagai Laporan Tugas Akhir \\
    di Bandung, pada tanggal \thedate{} \yearsidang{}.

    \vfill
    \begin{tabular}{c@{\hskip 0.6in}c}
        Pembimbing I, & Pembimbing II, \\
        \\
        \\
        \\
        \underline{\namapembimbing} & \underline{\namapembimbingkedua} \\
        NIP. \nippembimbing  & NIP. \nippembimbingkedua \\
    \end{tabular}
    

\end{center}
\clearpage

    
    \input{komponen-awal/pernyataan-orisinalitas}

    \pagestyle{plain}

    \clearpage
\chapter*{ABSTRAK}
\addcontentsline{toc}{chapter}{Abstrak}

% % Untuk keperluan perpus
% \thispagestyle{empty}
% \begin{center}
%     \textbf{\MakeUppercase{\thetitle}} \\
%     \vspace{5mm}
%     \small Oleh \\
%     \small \textbf{\theauthor} \\
%     \small \textbf{NIM: \nim} \\
%     \textbf{(Program Studi Sistem dan Teknologi Informasi)}
%     \vspace{5mm}
% \end{center}

Lorem ipsum dolor sit amet, consectetur adipiscing elit. Integer at justo condimentum quam fringilla egestas sed non dui. In at feugiat leo. Quisque maximus feugiat condimentum. Nunc molestie egestas ante vitae accumsan. Lorem ipsum dolor sit amet, consectetur adipiscing elit. Ut convallis at massa sed sagittis. Curabitur venenatis eleifend ipsum, sed sodales nibh mollis ut.


\vspace{15mm}
Kata kunci: lorem ipsum
\clearpage

    \clearpage
\chapter*{ABSTRACT}
\addcontentsline{toc}{chapter}{Abstract}


% Untuk keperluan perpus
% \thispagestyle{empty}
% \begin{center}
%     \textbf{\MakeUppercase{Machine Learning Implementation to Classify ITB Dropout Students using Academic Transcript}} \\
%     \vspace{5mm}
%     \small By \\
%     \small \textbf{\theauthor} \\
%     \small \textbf{NIM: \nim} \\
%     \textbf{(Information Systems and Technology Program)}
%     \vspace{5mm}
% \end{center}

Lorem ipsum dolor sit amet, consectetur adipiscing elit. Integer at justo condimentum quam fringilla egestas sed non dui. In at feugiat leo. Quisque maximus feugiat condimentum. Nunc molestie egestas ante vitae accumsan. Lorem ipsum dolor sit amet, consectetur adipiscing elit. Ut convallis at massa sed sagittis. Curabitur venenatis eleifend ipsum, sed sodales nibh mollis ut.


\vspace{15mm}
Key words: lorem ipsum

\clearpage

    \chapter*{Prakata}
\addcontentsline{toc}{chapter}{Prakata}

Prakata dibuat oleh penulis dengan muatan pernyataan syukur, latar belakang pembahasan masalah disertai tujuan singkat, hambatan yang dialami, bantuan yang diterima, ucapan terima kasih, keterbukaan menerima saran perbaikan, harapan penulis, dsb \autocite{penulisanilteks}.

Donec non nisl ac leo blandit fringilla sodales eget lorem. Curabitur ut aliquam sem. Phasellus iaculis tortor eros, id molestie erat pulvinar vel. Donec feugiat ante et ipsum faucibus vestibulum. Sed egestas tristique vestibulum. Suspendisse urna dolor, vulputate imperdiet lectus vel, malesuada finibus ante. Phasellus ornare, enim auctor molestie fermentum, diam elit sollicitudin augue, ac faucibus leo elit in ante. Nunc ante magna, auctor vel ex a, sodales tincidunt lectus. Donec vel tellus cursus, blandit ipsum at, luctus ligula. Quisque mi erat, rutrum in ex eu, molestie imperdiet turpis. Lorem ipsum dolor sit amet, consectetur adipiscing elit. Nunc eget hendrerit turpis. Curabitur in eros porta, convallis purus ut, sodales dui. Morbi porta dapibus elit. Fusce semper orci sagittis, facilisis odio a, commodo nibh. Maecenas non maximus nisi.

\vspace{15mm}
\begin{tabularx}{\textwidth}{l@{\hskip 0.635\textwidth}l}
    & Bandung, \thedate{} \yearsidang{}\\
    & Penulis
\end{tabularx}


    \titleformat*{\section}{\centering\bfseries\Large\MakeUpperCase}
	
	\clearpage
    \tableofcontents
    
    
    %----------------------------------------------------------------%
    % Daftar Gambar
    %----------------------------------------------------------------%
    \clearpage
    {%
		\let\oldnumberline\numberline%
		\renewcommand{\numberline}{\figurename~\oldnumberline}%
		\listoffigures%
	}
	
	
    %----------------------------------------------------------------%
    % Daftar Tabel
    %----------------------------------------------------------------%
	\clearpage
    {%
		\let\oldnumberline\numberline%
		\renewcommand{\numberline}{\tablename~\oldnumberline}%
		\listoftables%
	}

    %----------------------------------------------------------------%
    % Daftar Rumus
    % (Un-comment bila dibutuhkan)
    %----------------------------------------------------------------%
    % 	\clearpage
    %     {%
    %     	\let\oldnumberline\numberline%
    % 		\renewcommand{\numberline}{Rumus~\oldnumberline}%
    % 		\listofmyequations%
    % 	}
    
    %----------------------------------------------------------------%
    % Daftar Algoritme
    % (Un-comment bila dibutuhkan)
    %----------------------------------------------------------------%
    % 	\clearpage
    %     {%
    %     	\let\oldnumberline\numberline%
    % 		\renewcommand{\numberline}{\lstlistingname~\oldnumberline}%
    % 		\lstlistoflistings%
    % 	}
	
    
    \newpage
    \setcounter{savepage}{\arabic{page}}

    \titleformat*{\section}{\bfseries\Large}
    
    % Konfigurasi Bab
    \renewcommand{\chaptername}{BAB}
    \renewcommand{\thechapter}{\Roman{chapter}}
    \pagenumbering{bychapter}
    \setlength{\parindent}{1cm}

    %----------------------------------------------------------------%
    % Daftar Bab
    % (Bila butuh bab tambahan, buat file .tex baru dan input saja
    %----------------------------------------------------------------%
    \chapter{Pendahuluan}

\section{Latar Belakang}

Lorem ipsum dolor sit amet, consectetur adipiscing elit. Proin et maximus dolor. Integer malesuada rutrum luctus. Vivamus a sagittis nibh. Donec fermentum a nisl at vehicula. Vivamus leo arcu, vestibulum vitae nisi sed, consequat ullamcorper orci. Nunc auctor ante in sollicitudin tempus. Phasellus sit amet porttitor nulla. Proin dignissim finibus nunc, at semper orci vestibulum vulputate. Aliquam erat volutpat. Morbi hendrerit sapien quis eleifend scelerisque. Fusce faucibus arcu aliquam sollicitudin eleifend. Curabitur sit amet nisi interdum, semper massa et, congue tortor. Nulla iaculis ex a lorem placerat, eu rutrum augue posuere. Mauris ac mi sed nisl congue viverra ac ut augue. Sed eu lacinia ante. Sed elementum bibendum laoreet.

\section{Rumusan Masalah}

Lorem ipsum dolor sit amet, consectetur adipiscing elit. Proin et maximus dolor: 
\begin{enumerate}
    \item Nulla tristique eros ut dolor lobortis viverra.
    \item Sed euismod erat ut leo molestie, sed malesuada libero ultrices.
    \item Integer a nibh elementum, volutpat neque facilisis, rutrum lacus.
\end{enumerate}

\section{Tujuan}

Lorem ipsum dolor sit amet, consectetur adipiscing elit. Proin et maximus dolor: 
\begin{enumerate}
    \item Nulla tristique eros ut dolor lobortis viverra.
    \item Sed euismod erat ut leo molestie, sed malesuada libero ultrices.
    \item Integer a nibh elementum, volutpat neque facilisis, rutrum lacus.
\end{enumerate}

\section{Batasan Masalah}

Lorem ipsum dolor sit amet, consectetur adipiscing elit. Proin et maximus dolor: 
\begin{enumerate}
    \item Nulla tristique eros ut dolor lobortis viverra.
    \item Sed euismod erat ut leo molestie, sed malesuada libero ultrices.
    \item Integer a nibh elementum, volutpat neque facilisis, rutrum lacus.
\end{enumerate}

\section{Metodologi}

Lorem ipsum dolor sit amet, consectetur adipiscing elit. Proin et maximus dolor: 
\begin{enumerate}
    \item Nulla tristique eros ut dolor lobortis viverra.
    \item Sed euismod erat ut leo molestie, sed malesuada libero ultrices.
    \item Integer a nibh elementum, volutpat neque facilisis, rutrum lacus.
\end{enumerate}

\section{Sistematika Pembahasan}

Lorem ipsum dolor sit amet, consectetur adipiscing elit. Proin et maximus dolor: 
\begin{enumerate}[label=Bab \arabic*,itemindent=*]
	\item Pendahuluan\\
	Lorem ipsum dolor sit amet, consectetur adipiscing elit. Proin et maximus dolor. Integer malesuada rutrum luctus. Vivamus a sagittis nibh. Donec fermentum a nisl at vehicula. 
	\item Tinjauan Pustaka\\
	Lorem ipsum dolor sit amet, consectetur adipiscing elit. Proin et maximus dolor. Integer malesuada rutrum luctus. Vivamus a sagittis nibh. Donec fermentum a nisl at vehicula. 
    \item Metodologi\\
    Lorem ipsum dolor sit amet, consectetur adipiscing elit. Proin et maximus dolor. Integer malesuada rutrum luctus. Vivamus a sagittis nibh. Donec fermentum a nisl at vehicula. 
    \item Hasil dan Evaluasi\\
    Lorem ipsum dolor sit amet, consectetur adipiscing elit. Proin et maximus dolor. Integer malesuada rutrum luctus. Vivamus a sagittis nibh. Donec fermentum a nisl at vehicula. 
    \item Penutup\\
    Lorem ipsum dolor sit amet, consectetur adipiscing elit. Proin et maximus dolor. Integer malesuada rutrum luctus. Vivamus a sagittis nibh. Donec fermentum a nisl at vehicula. 
\end{enumerate}

    \chapter{Tinjauan Pustaka}

Lorem ipsum dolor sit amet, consectetur adipiscing elit. Integer at justo condimentum quam fringilla egestas sed non dui. In at feugiat leo. Quisque maximus feugiat condimentum. Nunc molestie egestas ante vitae accumsan. Lorem ipsum dolor sit amet, consectetur adipiscing elit. Ut convallis at massa sed sagittis. Curabitur venenatis eleifend ipsum, sed sodales nibh mollis ut.


\section{Dasar \LaTeX}

Struktur dasar perintah Latex adalah \verb|\commandname[]{}|. Suatu perintah dalam Latex menerima dua macam argumen, yaitu \textit{mandatory arguments} dan \textit{optional arguments}. Argumen wajib ditulis didalam kurung kurawal sementara argumen tambahan dalam kurung kotak. 

Untuk menyusun struktur dokumen, gunakan \verb|\chapter{}| untuk bab, \verb|\section{}| untuk sub bab, \verb|\subsection{}| untuk anak sub bab, \verb|\subsubsection{}| untuk level selanjutnya, dan seterusnya. Argumen yang dimasukan cukup nama bab atau sub bab tersebut saja. Penomoran akan dilakukan otomatis oleh program (termasuk juga penomoran untuk gambar dan tabel, jadi tidak perlu mengurus daftar isi, tabel, dan gambar). Sebagai contoh, struktur penulisan bab ini adalah: 

\begin{lstlisting}[caption={Contoh struktur bab}]
    \chapter{Tinjauan Pustaka}
        \section{Dasar \LaTeX}
            Bla bla bla bla bla
        \section{List}
            Bla bla bla bla bla
\end{lstlisting}

Untuk menulis cetak tebal dan miring, gunakan \verb|\textbf{}| dan \verb|\textit{}|. Untuk penulisan kode, gunakan \verb/\verb||/. Untuk perintah \texttt{verb}, argumen wajib tidak harus menggunakan kurung kurawal, bisa menggunakan simbol lain selama simbol tersebut tidak termasuk didalam argumennya. Untuk mencantumkan URL, gunakan \verb|\url{}|. Komentar menggunakan simbol \verb|%|.

\begin{table}[H]
    \centering
    \begin{tabular}{l  l}
        \toprule
        Hasil & Kode LaTeX \\
        \midrule
        \textbf{Lorem ipsum} dolor sit amet & \verb|\textbf{Lorem ipsum}| dolor sit amet \\
        \textit{Lorem ipsum} dolor sit amet & \verb|\textit{Lorem ipsum}| dolor sit amet \\
        \verb|Lorem ipsum| dolor sit amet & \verb[\verb|Lorem ipsum|[ dolor sit amet \\
         & \verb[%[ Lorem ipsum dolor sit amet \\
        \bottomrule
    \end{tabular}
\end{table}

\section{List}

Ada dua macam \textit{list} yang bisa dibuat, \textit{ordered} dan \textit{unordered} (seperti di HTML atau \textit{bullets and numbering} di Microsoft Word dan Google Docs). Untuk \textit{unordered lists} kodenya adalah: 

\begin{lstlisting}[caption={Kode Latex untuk \textit{unordered list}}]
    \begin{itemize}
        \item Nulla tristique eros ut dolor lobortis viverra.
        \item Sed euismod erat ut leo molestie, 
              sed malesuada libero ultrices.
    \end{itemize}
\end{lstlisting}

\begin{lstlisting}[caption={Kode Latex untuk \textit{ordered list}}]
    \begin{enumerate}
        \item Nulla tristique eros ut dolor lobortis viverra.
        \item Sed euismod erat ut leo molestie, 
              sed malesuada libero ultrices.
    \end{enumerate}
\end{lstlisting}


\section{Bibliography atau Daftar Pustaka}

Rujukan ditulis dengan menggunakan perintah \verb|\textcite{}| atau  \verb|\autocite{}|. Perintah pertama digunakan bila mau mengacu ke nama penulis pada awal kalimat, sementara perintah kedua dapat digunakan bila ditulis pada akhir kalimat, misalkan: 

\begin{displayquote}
    "\textcite{penulisanilteks} mengatakan bahwa tata karya tulis ilmiah merupakan cara menyusun tulisan tentang perencanaan, pelaksanaan, dan hasil suatu kajian ilmiah"
    
    "Tata karya tulis ilmiah merupakan cara menyusun tulisan tentang perencanaan, pelaksanaan, dan hasil suatu kajian ilmiah \autocite{penulisanilteks}."
\end{displayquote}

Argumen yang diberikan bergantung pada entri didalam file \texttt{.bib} yang digunakan. Terdapat beberapa macam entri, misalnya \verb|@book| atau \verb|@article|. Bila mencari sumber dari \textit{Google Scholar}, entri dapat langsung didapatkan dengan menekan tombol kutip, kemudian menekan tombol "BibTeX". Isi didalam file \texttt{.bib} tidak perlu diurutkan, daftar pustaka akan otomatis diurutkan oleh program. 

\begin{lstlisting}[caption={Contoh entri dalam BibTeX}]
    @book{penulisanilteks,
      author    = {{Tim Dosen Tata Tulis Karya Ilmiah ITB}}, 
      title     = {Metode Penulisan Ilteks},
      year      = 2017,
    }
\end{lstlisting}

\section{Gambar}

\textit{Gambar \ref{gambar:logo_itb} merupakan logo Institut Teknologi Bandung. Terdapat beberapa jenis variasi logo yang dapat digunakan, tergantung tujuan penggunaan logo. Contoh variasi lain logo ITB dapat diunduh pada halaman \url{https://ditsti.itb.ac.id/download-logo-itb/}.}

\begin{figure}[H]
    \centering
  	\includegraphics[width=0.38\textwidth]{aset/logo-itb.jpg}
  	\caption{Logo ITB}
  	\label{gambar:logo_itb}
\end{figure}

\begin{lstlisting}[caption={Kode Latex untuk mencantumkan gambar}, float]
    \begin{figure}[h]
        \centering
      	\includegraphics[]{}
      	\caption{}
      	\label{}
    \end{figure}
\end{lstlisting}

Untuk mencantumkan gambar, beberapa hal yang perlu diperhatikan adalah sebagai berikut:

\begin{enumerate}
    \item Gambar harus diberikan nama. Nama gambar ditambahkan dengan menggunakan perintah \verb|\caption{}|. Nama gambar dicantumkan dibawah gambar.
    \item Gunakan \textit{title case} untuk nama gambar (serta tabel, rumus, dsb).
    \item Gambar harus dirujuk dalam paragraf pembahasan. Untuk merujuk kepada gambar, gunakan perintah \verb|ref{}|. 
    \item Ketika mencantumkan gambar yang dibuat oleh orang lain, termasuk misalnya mengambil gambar dari internet, harus dituliskan sumber kutipan. Hal ini untuk menghindari tindakan penjiplakan. 
\end{enumerate}

\section{Tabel}

\textit{Tabel \ref{tabel:sampel_dataset} merupakan tabel nilai mahasiswa dari prodi Sistem dan Teknologi Informasi pada beberapa mata kuliah semester ketiga dan sebelumnya. Setiap kolom merepresentasikan suatu mata kuliah dan setiap baris merupakan seorang mahasiswa. Nilai NaN berarti data tidak tersedia.}

\begin{table}[H]
    \centering
    \caption{Sampel dataset}
    \label{tabel:sampel_dataset}
    \begin{tabular}{rrrrrrrrr}
     \toprule
     EL2142 & IF1210 & II2110 & KI1202 & KU1072 & II2111 & II2110 & ... & TI3005\\
     \midrule
     4.0 & 3.0 & 4.0 & 4.0 & 4.0 & 4.0 & 4.0 & ... & NaN \\
     0.0 & 1.0 & NaN & 0.0 & 3.0 & NaN & NaN & ... & NaN \\
     4.0 & 3.5 & 3.0 & 4.0 & 3.0 & 3.0 & 3.0 & ... & 4.0 \\
     3.0 & 3.0 & 4.0 & 3.0 & NaN & 4.0 & 4.0 & ... & 3.0 \\
     3.0 & 2.5 & 4.0 & 3.0 & 3.0 & 4.0 & 4.0 & ... & 3.0 \\
     0.0 & 0.0 & 2.0 & 0.0 & 3.0 & 2.0 & 2.0 & ... & 3.0 \\
     NaN & 2.0 & 4.0 & 2.0 & 3.0 & 4.0 & 4.0 & ... & NaN \\
     4.0 & 3.5 & 3.5 & 3.0 & 4.0 & 3.5 & 3.5 & ... & 3.0 \\
     NaN & 2.0 & 4.0 & 3.5 & 2.5 & 4.0 & 4.0 & ... & 3.0 \\
     1.0 & 2.0 & 3.5 & 2.0 & 3.0 & 3.5 & 3.5 & ... & NaN \\
     \bottomrule
    \end{tabular}
\end{table}

Dalam menyusun tabel, jumlah dan \textit{alignment} kolom harus ditentukan terlebih dulu dengan memberikan argumen kedalam perintah \verb|\tabular| dan \verb|\longtable|. Dalam contoh yang diberikan, argumen yang diberikan adalah \verb|{rrr}|. Argumen tersebut berarti 3 kolom yang setiap kolomnya menggunakan rata kanan. Untuk rata tengah, gunakan \texttt{c} dan untuk rata kiri, gunakan \texttt{l}. Jadi, misalkan mau menyusun tabel dengan 5 kolom dengan rata kiri kecuali kolom pertama menggunakan rata tengah, argumen yang digunakan adalah \verb|{cllll}|. Untuk contoh tabel diatas yang digunakan adalah \verb|{rrrrrrrrr}|. Lebar kolom akan diatur otomatis oleh program. 

Setiap baris tabel diakhiri dengan \verb|\\| dan setiap sel dipisah menggunakan simbol \&. Jadi misalkan satu baris yang terdiri dari 3 kolom dengan nilai 1, 2, dan 3, penulisannya adalah \verb|1 & 2 & 3 \\|.

\begin{lstlisting}[caption={Kode Latex untuk tabel biasa}]
    \begin{table}[H]
        \centering
        \caption{}
        \label{}
        
        \begin{tabular}{rrr}
         \toprule
         Judul Kolom 1 & Judul Kolom 2 & Judul Kolom 3 \\
         \midrule
         a & b & c \\
         d & e & f \\
         g & h & i \\
         \bottomrule
        \end{tabular}

    \end{table}
\end{lstlisting}

Berbeda dengan \texttt{tabular}, \texttt{longtable} memiliki perintah tambahan seperti \verb|\endhead| dan \verb|\endfirsthead|. Gunakan \verb|\endfirsthead| untuk mendeklarasikan batas kepala tabel yang pertama kali muncul, sehingga caption dan tabel hanya perlu dideklarasikan disini.

\begin{lstlisting}[caption={Kode Latex untuk tabel yang melebihi 1 halaman}]
    \begin{longtable}{rrr}
        \caption{}
        \label{}\\
        \toprule
        Judul Kolom 1 & Judul Kolom 2 & Judul Kolom 3 \\
        \midrule
        \endfirsthead
    
        \toprule
        Judul Kolom 1 & Judul Kolom 2 & Judul Kolom 3 \\
        \midrule
        \endhead
        
        a & b & c \\
        d & e & f \\
        g & h & i \\
        \bottomrule
    \end{longtable}
\end{lstlisting}

Untuk mencantumkan tabel, beberapa hal yang perlu diperhatikan adalah sebagai berikut:

\begin{enumerate}
    \item Berikan nama dan label untuk tabel. Cara menuliskan nama untuk tabel sama seperti gambar, namun nama tabel dicantumkan diatas tabel. 
    \item Ada 2 \textit{package} yang dapat digunakan, yaitu \texttt{tabular} dan \texttt{longtable}. Untuk tabel yang tidak melebihi satu halaman, cukup gunakan \texttt{tabular} saja.
    \item Untuk \texttt{longtable}, setelah menuliskan label, harus diakhiri dengan \verb|\\|.
\end{enumerate}

\section{Rumus}

\textit{Salah satu rumus untuk menghitung jarak yang umum digunakan adalah euclidean distance (rumus \ref{persamaan:euclidean_distance}).}

\begin{equation}
    \label{persamaan:euclidean_distance}
    d_{(x,y)} = \sqrt{\sum_{i=1}^{k} (x_i - y_i)^2}
\end{equation}
\myequations{Euclidean Distance}

Dalam template ini, untuk menyusun daftar rumus, digunakan perintah \verb|\myequations{}|. Cara kerjanya mirip dengan \verb|\caption{}|. Bila tidak akan menyusun daftar rumus, perintah tersebut tidak usah digunakan.

\begin{lstlisting}[caption={Kode Latex untuk rumus}]
    \begin{equation}
        \label{}
        % Tulis persamaan disini 
    \end{equation}
    \myequations{} % Tidak perlu bila tidak akan menyusun daftar rumus
\end{lstlisting}

\section{Algoritme}

\begin{lstlisting}[caption={Kode Latex untuk algoritme}]
    \begin{lstlisting}[caption={}, float]
        % Tulis kode disini
    \end{lstlistingg} % hapus typonya (lebih 1 huruf g)
\end{lstlisting}
    \chapter{Metodologi}

Lorem ipsum dolor sit amet, consectetur adipiscing elit. Integer at justo condimentum quam fringilla egestas sed non dui. In at feugiat leo. Quisque maximus feugiat condimentum. Nunc molestie egestas ante vitae accumsan. Lorem ipsum dolor sit amet, consectetur adipiscing elit. Ut convallis at massa sed sagittis. Curabitur venenatis eleifend ipsum, sed sodales nibh mollis ut.


\section{Tinjauan Pustaka}

\begin{enumerate}
    \item Beberapa teman terbaik untuk mencari referensi adalah \href{https://scholar.google.com}{Google Scholar} dan VPN ITB yang membuka akses ke beberapa jurnal secara resmi (selengkapnya di \url{https://lib.itb.ac.id/e-journal}).
    \item Buat catatan (bisa ditulis atau pakai \textit{spreadsheet}) untuk merangkum referensi yang dibaca, terutama terkait penelitian serupa.
    \item Perhatikan referensi yang digunakan, terutama kapan referensi tersebut dilakukan/dipublikasi dan sumbernya. Jangan gunakan sumber seperti Wikipedia atau Blog. Namun, Wikipedia bisa digunakan untuk melihat referensi yang bisa digunakan (cek ke bagian bawah suatu artikel, lihat referensinya).
    \item Utamakan penjelasan penelitian-penelitian serupa. Pembahasan teori atau algoritme misalnya \textit{convolutional neural network} atau \textit{blockchain} tidak usah terlalu panjang.
    \item Ketika membaca paper, baca dulu abstrak dan kesimpulannya. Bila cocok atau dirasa paper tersebut bisa dijadikan referensi, baru baca dengan menyeluruh.
    \item Untuk kode program, beberapa modul biasanya meminta untuk dikutip (contoh: scikit-learn). Tunjukan rasa terima kasih dengan setidaknya mengutip pembuatanya (Biasanya ada entri BibTeX di GitHubnya).
\end{enumerate}


\section{Tata Kata dan Kalimat}

\begin{enumerate}
    \item Gunakan kata yang baku, cek KBBI kalau kurang yakin.
    \item Gunakan cetak miring untuk kata dalam bahasa asing.
    \item Gunakan kalimat lengkap. Kalimat lengkap adalah kalimat yang memiliki Subjek dan Predikat.
    \item Pisahkan kalimat yang terlalu panjang menjadi beberapa kalimat pendek yang disambung dengan kata sambung (dengan demikian, oleh karena itu, akan tetapi, namun, dsb.)
\end{enumerate}

\section{Tata Paragraf}

\begin{enumerate}
    \item Ingat kalau satu paragraf terbentuk dari satu pikiran utama (PU) disertai beberapa pikiran penjelas (PP).
\end{enumerate}

    \chapter{Hasil}

Lorem ipsum dolor sit amet, consectetur adipiscing elit. Integer at justo condimentum quam fringilla egestas sed non dui. In at feugiat leo. Quisque maximus feugiat condimentum. Nunc molestie egestas ante vitae accumsan. Lorem ipsum dolor sit amet, consectetur adipiscing elit. Ut convallis at massa sed sagittis. Curabitur venenatis eleifend ipsum, sed sodales nibh mollis ut.
    \input{komponen-utama/bab-5}
    
    %----------------------------------------------------------------%

    % Konfigurasi Daftar Pustaka
    \begin{flushleft}
    	\clearpage
    	\pagenumbering{roman}
    	\setcounter{page}{\thesavepage}
    	\addcontentsline{toc}{chapter}{Daftar Pustaka}
    	\printbibliography[title=Daftar Pustaka]
    \end{flushleft}
    
    % Konfigurasi Lampiran
    \clearpage
    \appendix
    \addtocontents{toc}{\protect\renewcommand{\protect\cftchappresnum}{Lampiran }}
    \addtocontents{toc}{\protect\renewcommand{\protect\cftchapnumwidth}{6em}}
    \renewcommand{\chaptername}{Lampiran}
    \pagenumbering{bychapter}
    
    %----------------------------------------------------------------%
    % Daftar Lampiran
    % (Bila butuh lampiran tambahan, buat file .tex baru dan input saja
    %----------------------------------------------------------------%
    \chapter{Contoh Lampiran}

Cantumkan lampiran disini. Setiap gambar, tabel, atau kode sebaiknya dijadikan satu lampiran terpisah dan diberikan nama. 
    
    %----------------------------------------------------------------%
    
\end{document}
