\chapter{Metodologi}

Lorem ipsum dolor sit amet, consectetur adipiscing elit. Integer at justo condimentum quam fringilla egestas sed non dui. In at feugiat leo. Quisque maximus feugiat condimentum. Nunc molestie egestas ante vitae accumsan. Lorem ipsum dolor sit amet, consectetur adipiscing elit. Ut convallis at massa sed sagittis. Curabitur venenatis eleifend ipsum, sed sodales nibh mollis ut.


\section{Tinjauan Pustaka}

\begin{enumerate}
    \item Beberapa teman terbaik untuk mencari referensi adalah \href{https://scholar.google.com}{Google Scholar} dan VPN ITB yang membuka akses ke beberapa jurnal secara resmi (selengkapnya di \url{https://lib.itb.ac.id/e-journal}).
    \item Buat catatan (bisa ditulis atau pakai \textit{spreadsheet}) untuk merangkum referensi yang dibaca, terutama terkait penelitian serupa.
    \item Perhatikan referensi yang digunakan, terutama kapan referensi tersebut dilakukan/dipublikasi dan sumbernya. Jangan gunakan sumber seperti Wikipedia atau Blog. Namun, Wikipedia bisa digunakan untuk melihat referensi yang bisa digunakan (cek ke bagian bawah suatu artikel, lihat referensinya).
    \item Utamakan penjelasan penelitian-penelitian serupa. Pembahasan teori atau algoritme misalnya \textit{convolutional neural network} atau \textit{blockchain} tidak usah terlalu panjang.
    \item Ketika membaca paper, baca dulu abstrak dan kesimpulannya. Bila cocok atau dirasa paper tersebut bisa dijadikan referensi, baru baca dengan menyeluruh.
    \item Untuk kode program, beberapa modul biasanya meminta untuk dikutip (contoh: scikit-learn). Tunjukan rasa terima kasih dengan setidaknya mengutip pembuatanya (Biasanya ada entri BibTeX di GitHubnya).
\end{enumerate}


\section{Tata Kata dan Kalimat}

\begin{enumerate}
    \item Gunakan kata yang baku, cek KBBI kalau kurang yakin.
    \item Gunakan cetak miring untuk kata dalam bahasa asing.
    \item Gunakan kalimat lengkap. Kalimat lengkap adalah kalimat yang memiliki Subjek dan Predikat.
    \item Pisahkan kalimat yang terlalu panjang menjadi beberapa kalimat pendek yang disambung dengan kata sambung (dengan demikian, oleh karena itu, akan tetapi, namun, dsb.)
\end{enumerate}

\section{Tata Paragraf}

\begin{enumerate}
    \item Ingat kalau satu paragraf terbentuk dari satu pikiran utama (PU) disertai beberapa pikiran penjelas (PP).
\end{enumerate}
